\documentclass[11pt,a4paper]{article}
\usepackage[utf8]{inputenc}
\usepackage[english]{babel}
\usepackage[left=2cm,right=2cm,top=3cm,bottom=4cm]{geometry}
\usepackage{amsmath}
\usepackage{mathtools}
\usepackage{cases}


\author{Ari-Pekka Honkanen}
\title{PyTTE: The Technical Document}
\begin{document}
\maketitle
\section{Introduction}
PyTTE (pronounced \emph{pie-tee-tee-ee}) is a Python package for solving X-ray diffraction curves of bent crystals in reflection and transmission geometries. The computation of the diffraction curves is based on the numerical integration of 1D Takagi-Taupin equation (TTE) which is derived from a more general Takagi-Taupin theory describing the propagation of electromagnetic waves in a (quasi)periodic medium. Both energy and angle scans are supported.
This document describes the theoretical basis behind PyTTE.

\section{Takagi-Taupin equation}
In the typical two-beam case, the Takagi-Taupin equations are can be written as
\begin{subnumcases}{}
\frac{\partial D_0}{\partial s_0} =  i c_0 D_0 + i c_{\bar{h}} D_h \label{eq:TT_typicala} \\
\frac{\partial D_h}{\partial s_h} =  i \left(c_0 + \beta + \frac{\partial (\mathbf{h}\cdot\mathbf{u}) }{\partial s_h}  \right) D_h
+ ic_h D_0, \label{eq:TT_typicalb}
\end{subnumcases}
where $D_0$ and $D_h$ are the pseudoamplitudes of the incident and diffracted waves, and $s_0$ and $s_h$ coordinates along their direction of propagation, respectively. The deformation of the crystal is contained in $\mathbf{h}\cdot\mathbf{u}$ which will be considered in detail later. The coefficients $c_{0,h,\bar{h}}$ are given by
\begin{equation}
c_0 = \frac{k \chi_0}{2} \qquad c_{h,\bar{h}} = \frac{k C \chi_{h,\bar{h}}}{2},
\end{equation} 
where $k=2 \pi/\lambda$ and $C = 1$ for $\sigma$-polarization and $\cos 2 \theta_B$ for $\pi$-polarization. The deviation parameter $\beta = (k_h^2 - k_0^2)/2k_h$ is quite often approximated by $\beta \approx k \Delta \theta \sin 2 \theta_B$, where $\Delta \theta$ is the deviation from the Bragg angle. However, since this approximation ceases to be valid near the backscattering condition, PyTTE uses a more general form
\begin{equation}
\beta = \frac{2 \pi}{d_h}\left(\frac{\lambda}{2 d_h}-\sin \theta \right) = h \left(\frac{\lambda}{2 d_h}-\sin \theta \right),
\end{equation}
where $d_h$ is the interplanar separation of the diffractive planes corresponding to the reciprocal vector $\mathbf{h}$ and $\theta$ is the incidence angle relative to the aforementioned planes.

%THIS PARAGRAPH NEEDS A FIGURE
The partial derivatives with respect to $s_0$ and $s_h$ can be written in Cartesian coordinates as
\begin{equation}
\frac{\partial}{\partial s_0} = \cos \alpha_0 \frac{\partial}{\partial x} - \sin \alpha_0 \frac{\partial}{\partial z} \qquad \frac{\partial}{\partial s_h} = \cos \alpha_h \frac{\partial}{\partial x} + \sin \alpha_h \frac{\partial}{\partial z}.
\end{equation}
The incidence and exit angles $\alpha_0$ and $\alpha_h$ with respect to the crystal surface are related to $\theta$ by $\alpha_0 = \theta + \varphi$ and $\alpha_h = \theta - \varphi$, where $\varphi$ is the asymmetry angle (positive clockwise). When seeking a solely depth-dependent solution for $D_0$ and $D_h$, we may drop the $x$-derivatives and thus obtain 
\begin{subnumcases}{}
\frac{d D_0}{d z} =  - i \gamma_0 c_0 D_0 - i \gamma_0 c_{\bar{h}} D_h \label{eq:TT_b} \\
\frac{d D_h}{d z} =  i \gamma_h \left(c_0 + \beta + \frac{\partial (\mathbf{h}\cdot\mathbf{u}) }{\partial s_h}  \right) D_h + i \gamma_h c_h D_0, 
\end{subnumcases}
where $\gamma_0 = 1/\sin \alpha_0$ and $\gamma_h = 1/\sin \alpha_h$. By defining $\xi = D_h/D_0$, the equations can be written as a single ordinary differential equation
\begin{equation}
\frac{d \xi}{d z} = i \gamma_0 c_{\bar{h}} \xi^2 + i \left[ (\gamma_0+\gamma_h)c_0 + \gamma_h \beta + \gamma_h \frac{\partial (\mathbf{h}\cdot\mathbf{u}) }{\partial s_h} \right] \xi + i \gamma_h c_h
\label{eq:xi}
\end{equation}

In the reflection geometry (\emph{i.e.} the Bragg case), the reflectivity $R$ of the crystal with thickness $t$ can be solved by integrating Equation~\eqref{eq:xi} from the bottom of the crystal $z=-t$ to the top surface $z=0$. The initial condition is set $\xi(-t)=0$. The reflectivity, which is defined in the terms of integrated intensity, is then computed $R = \gamma_0/\gamma_h|\xi(0)|^2$ where $\gamma_0/\gamma_h$ takes care of different footprint sizes in the asymmetric case. Using the solved $\xi$ and Equation~\eqref{eq:TT_b}, the transmission $T=|D_0(-t)/D_0(0)|^2$ can be then solved from
\begin{equation}
\frac{d D_0}{d z} =  - i \left( \gamma_0 c_0  + \gamma_0 c_{\bar{h}} \xi \right) D_0
\end{equation}
by integrating from $z=0$ to $z=-t$. The transmission geometry (the Laue case) is more straightforward, as the ODEs for $\xi$ and $D_0$ can be integrated simultaneously from $z=0$ to $z=-t$. With the initial conditions are set $\xi(0) = 0$ and $D_0(0)=1$, the forward-diffracted intensity at the exit surface is $|D_0(-t)|^2$ and the diffracted intensity is $|D_h(-t)|^2=|\xi(-t)D_0(-t)|^2$.

\section{Deformation}
As stated in the previous section, the deformation is introduced through $\partial_h(\mathbf{h}\cdot \mathbf{u})$ term where $\mathbf{u}$ is the displacement vector field. Taking the asymmetry into account, the reciprocal vector is given by $\mathbf{h} = h \sin \varphi \hat{\mathbf{x}} + h \cos \varphi \hat{\mathbf{z}}$. Thus
\begin{equation}
\frac{\partial (\mathbf{h}\cdot \mathbf{u})}{\partial s_h} = h \sin \varphi \frac{\partial u_x}{\partial s_h} + h \cos \varphi \frac{\partial u_z}{\partial s_h}.
\end{equation}
Again we write the partial derivatives in terms of $x$ and $z$. In this case, however, neither $x$- or $z$-derivatives can be dropped as they both contain physical information about the rotation and the separation of the diffractive planes. Since the beam propagates also in the $x$-direction, the situation is not strictly speaking one dimensional. However, since the $x$-coordinate is geometrically related to $z$, the problem can be treated as such. Therefore the deformation term becomes
\begin{equation}
\frac{\partial (\mathbf{h}\cdot \mathbf{u})}{\partial s_h} = h \left( 
\sin \varphi \cos \alpha' \frac{\partial u_x}{\partial x} 
+\sin \varphi \sin \alpha' \frac{\partial u_x}{\partial z} 
+\cos \varphi \cos \alpha' \frac{\partial u_z}{\partial x} 
+\cos \varphi \sin \alpha' \frac{\partial u_z}{\partial z} 
 \right),
\end{equation}
where the derivatives, that are functions of $x$ and $z$, are made only $z$-dependent with $x(z)=-z \cot \alpha$. PyTTE computes the strain term from the Jacobian of $\mathbf{u}$.

\subsection{Anisotropic plate}
According to \cite{Sanchez_del_Rio_2015}, the components of the displacement field for an anisotropic plate bent by two (scaled) torques $m_x$ and $m_y$ are
\begin{align}
u_x &= (S_{11} m_x + S_{12} m_y) x z + (S_{51} m_x + S_{52} m_y)\frac{z^2}{2} + (S_{61} m_x +S_{62} m_y) \frac{y z}{2} \\
u_y &= (S_{21} m_x + S_{22} m_y) y z + (S_{41} m_x + S_{42} m_y)\frac{z^2}{2} + (S_{61} m_x +S_{62} m_y) \frac{x z}{2} \\
u_z &= -(S_{11} m_x + S_{12} m_y)\frac{x^2}{2} -(S_{21} m_x + S_{22} m_y)\frac{y^2}{2} -(S_{61} m_x +S_{62} m_y) \frac{x y}{2} +(S_{31} m_x + S_{32} m_y)\frac{z^2}{2},
\end{align} 
where $S_{ij}$ are the components of the compliance matrix. Thus we find the partial derivatives:
\begin{align}
\frac{\partial u_x}{\partial x} &= (S_{11} m_x + S_{12} m_y) z \\
\frac{\partial u_x}{\partial z} &= (S_{11} m_x + S_{12} m_y) x + (S_{51} m_x + S_{52} m_y) z 
+ (S_{61} m_x + S_{62} m_y) \frac{y}{2}\\
\frac{\partial u_z}{\partial x} &= -(S_{11} m_x + S_{12} m_y) x -(S_{61} m_x + S_{62} m_y) \frac{y}{2} \\
\frac{\partial u_z}{\partial z} &= (S_{31} m_x + S_{32} m_y) z
\end{align} 
The torques are related to the bending radii $R_i$ by\footnote{In this work the signs of $R_i$ is the opposite to that of \cite{Sanchez_del_Rio_2015}}
\begin{equation}
\frac{1}{R_x} = -S_{11} m_x - S_{12} m_y, \qquad
\frac{1}{R_y} = -S_{21} m_x - S_{22} m_y.
\end{equation}
Thus
\begin{equation}
m_x = \frac{1}{S_{11}S_{22}-S_{12}S_{21}}\left(\frac{S_{12}}{R_y}-\frac{S_{22}}{R_x} \right), \qquad
m_y = \frac{1}{S_{11}S_{22}-S_{12}S_{21}}\left(\frac{S_{21}}{R_x}-\frac{S_{11}}{R_y} \right).
\label{eq:ms}
\end{equation}
\subsection{Isotropic plate}
The general anisotropic equations simplify considerably when assuming the plate to be isotropic. The isotropic compliance matrix is given by
\begin{equation}
S = \frac{1}{E}\left[\begin{matrix}
1 & -\nu & -\nu & 0 & 0 & 0 \\
-\nu & 1 & -\nu & 0 & 0 & 0 \\
-\nu & -\nu & 1 & 0 & 0 & 0 \\
0 & 0 & 0 & 2(1+\nu) & 0 & 0 \\
0 & 0 & 0 & 0 &2(1+\nu)  & 0 \\
0 & 0 & 0 & 0 & 0 & 2(1+\nu) \\
\end{matrix}\right],
\end{equation}
where $E$ is Young's modulus and $\nu$ is the Poisson ratio.
Plugging these into \eqref{eq:ms}, we obtain
\begin{equation}
m_x = -\frac{E}{1-\nu^2}\left(\frac{1}{R_x}+\frac{\nu}{R_y} \right), \qquad
m_y = -\frac{E}{1-\nu^2}\left(\frac{1}{R_y}+\frac{\nu}{R_x} \right)
\end{equation}
and thus 
\begin{equation}
\frac{\partial u_x}{\partial x} = -\frac{z}{R_x}, \qquad
\frac{\partial u_x}{\partial z} = -\frac{x}{R_x}, \qquad
\frac{\partial u_z}{\partial x} = \frac{x}{R_x}, \qquad
\frac{\partial u_z}{\partial z} = \frac{\nu}{1-\nu}\left(\frac{1}{R_x}+\frac{1}{R_y}\right)z.
\end{equation}
Finally, since PyTTE assumes the top surface is at $z=0$, we need to shift the $z$-axis $z \rightarrow z + t/2 $
\bibliographystyle{unsrt}
\bibliography{documentation}
\end{document}